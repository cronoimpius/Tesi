\chapter{Gestione dati eterogenei con MLP}
\label{ch:MLP}
\section{Gestione dei dati}
I risultati ottenuti con la CNN possono essere migliorati o resi più attendibili con l'utilizzo
di altre tipologie di dati. Questi sono le informazioni relative al paziente al momento dell'arrivo in 
ospedale.
\\\\
Tali dati sono di vario tipo, quelli usati dal dataset considerato sono: 
\begin{itemize}
    \item Categorico, esprimono l'appartenenza ad una o più categorie
    \item Booleano, esprimo se il paziente è affetto da una certa patologia o presenta alcuni 
    \item Numerici
\end{itemize}
In questa fase è importante che tutti i dati siano presenti per ogni paziente. Per tale motivo si è dovuto 
trovare un sottinsieme del dataset che presenti il maggior numero di dati. 
\\\\
Per fare ciò sono state eliminate le colonne che non avevano corrispondenze 
tra i pazienti, ovvero erano vuote oppure presentavano dati solo per pochi pazienti.
Al fine di ottenere un risultato migliore sono state eliminati anche i pazienti che non presentavano dati per 
la maggiorparte delle colonne del dataset.
\\\\
Effettuata tale operazione, si è proceduto col creare un unico dataset ottenuto dall'unione 
di training set e test set. Per fare ciò le colonne presenti tra i due devono essere le stesse.
Per tale motivo si è giunti ad un dataset formato dalle seguenti colonne:
\begin{itemize}
    \item Ospedale, dato categorico che rappresenta l'ospedale che ha accolto il paziente tra A,B,C,D,E e F 
    \item Età, dato numerico
    \item Sesso, categorico binario, ovvero maschio o femmina
    \item Tosse, binario
    \item Difficoltà respiratorie
    \item Numero di cellule bianche, dato numerico che indica la percentuale di globuli bianchi nel sangue
    \item Pressione sanguigna alta,  binario
    \item Diabete, binario
    \item Demenza, binario 
    \item BPCO (Broncopneumopatia cronica ostruttiva), binario 
    \item Cancro, binario 
    \item Malattia renale cronica, binario
\end{itemize}
Tali dati sono presenti per 946 pazienti del training set e 472 del test set, per cui abbiamo un dataset di 1218 pazienti.
Questo dataset sarà poi suddiviso in training, validation e test set, per effettuare l'allenamento della rete e per verificare la capacità di effettuare previsioni.
\\\\
Ora che si presenta il dataset completo, si procede col trasformare tutti i dati nello stesso formato, ovvero in valori binari.
Partendo dai dati categorici, si procede usando la codifica one-hot. Tale procedura prevede che le categorie relative al dato vengano 
trasformate in una rappresentazione binaria, in cui ogni categoria è rappresentata da una serie di zeri ed un unico uno presente nella categoria
codificata.
Per quanto riguarda il sesso, essendo una categoria binaria, si necessida di soli due bit per rappresentarla, ovvero
una sarà rappresentata da 10 e l'altra da 01.
L'ospedale è un dato che prevede sei categorie, per cui queste saranno rappresentate da sei bit. Ogni rappresentazione
presenterà cinque zeri ed un unico uno (es. 001000).
\\\\
A livello pratico tali trasfromazioni sono state implementate usando la libreria scikit-learn, in particolare le funzioni
MultiLabelBinarizer(), per l'ospedale, e LabelBinarizer() per il sesso. In tal modo si riesce a trasformare le categorie in array di 
bit, ovvero valori compresi in [0,1].
\\\\
Per gestire i valori numerici si sfrutta un'altra funzione di scikit-learn, ovvero MinMaxScaler().
Tale funzione scala i valori dati in input in un range specificato, nel caso in considerazione [0,1].
La trasformazione avviene mediante:
\begin{equation*}
    \begin{array}{l}
        X_{std} = (X - X.min(axis=0)) / (X.max(axis=0) - X.min(axis=0)) \\
        X_{scaled} = X_{std} * (max - min) + min
    \end{array}
\end{equation*}
\\\\
I valori binari infine sono rimasti invariati.

\section{Struttura della rete}