\chapter{Implementazione: CNN}
\label{ch:CNN}

Lo sviluppo della rete neurale convoluzionale, per effettuare previsioni partendo dalle immagini, è basata sulla rete 
MobileNetV2. L'uso di tale rete come base consente di usare dei pesi preallenati in modo da rendere il processo di training sul dataset più efficiente.
\\\\
Tuttavia tale processo necessita di alcune fasi aggiuntive per fare in modo che la rete riesca a prendere i dati della dimensione esatta 
e restituire delle previsioni nell'insieme desiderato.
\\\\
Tali operazioni sono:
\begin{itemize}
    \item Inserire la dimensione delle immagini nel layer di input
    \item Effettuare fine tuning e transfer learning per poter usare i pesi preallenati 
    \item Inserire dei layer finali per ottenere degli output significativi
\end{itemize}

La dimensione scelta delle immagini è (224x224), dunque la rete prende in ingresso degli array di dimensione 224x224x3 (dove quest'ultimo indica l'uso dei colori RGB).
Per tale motivo il layer di input deve accettare tale dimensione.
\\\\
I pesi preallenati che si sono scelti sono pesi allenati su ImageNet. ImageNet è un dataset di immagini suddivise in 1000 classi. Il dataset su cui si deve svolgere il training, tuttavia, possiede 
unicamente una classe, per via del fatto che abbiamo trasformato il valore della prognosi da una string ad un valore binario. 
Per cui l'immagine può appartenere o meno a tale classe.
\\\\
Al fine di poter usare tali pesi per effettuare il training, si necessita dunque di ulteriori passaggi: 
\begin{itemize}
    \item Transfer Learning
    \item Fine tuning
\end{itemize}

\section{Transfer Learning}

\section{Fine Tuning}