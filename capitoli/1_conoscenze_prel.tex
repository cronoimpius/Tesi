\chapter{Conoscenze preliminari}
\label{ch:Conoscenze Preliminari}
\section{Strumenti usati}
Esistono vari framework per lo sviluppo di reti neurali, uno dei più usati è TensorFlow \cite{tsf}. 
\\
TensorFlow presenta modelli già creati e allenati, i cui pesi possono essere presi per sfruttare tale modello.
Insieme all'uso di altre librerie come Keras l'implementazione di una rete neurale che gestisca i dati a disposizione 
è molto semplificato.
\\
La rete di base usata per partire con l'analisi delle immagini è la MobileNetV2. Tale rete è una CNN che, prendendo come input delle 
immagini.
A tale rete, tuttavia, sono state apportate delle modifiche per via delle dimensioni dell'input e della tipologia di dato 
che si vuole ottenere come previsione.
\\
Oltre a questa tipologia di rete si è sfruttata anche una U-Net, al fine di omogeneizzare tutte le immagini che vengono date come input della MobileNetV2.
La U-Net usata è EfficientNet, con dei pesi preallenati per riconoscere immagini simili a quelle presenti nel dataset di riferimento.

