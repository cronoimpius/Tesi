\chapter{Dataset}
\label{ch:Dataset}
\section{Composizione}
*inserire link all'hackaton e immagini dataset*
\\\\
Il dataset usato per effettuare lo studio è stato preso da un hackaton riguardante la creazione di una rete neurale in grado di apprendere 
da immagini relative a radiografie di pazienti affetti da covid in modo da predirre la gravità della prognosi.
\\\\
Dunque il dataset usato è formato da immagini raffiguranti radiografie dei polmoni, le quali tuttavia non sono omogenee. Per tale motivo 
si è dovuto effettuare un passaggio preliminare in modo da rendere le immagini tutte della stessa dimensione, ma non distorcendole.
\\\\
Oltre alle immagini sono presenti anche un insieme di metadati riguardanti lo stato in cui il paziente è entrato all'ospedale e l'anamnesi dello stesso.
Tali metadati presentano descrivono varie informazioni relative al paziente, per tale motivo sono rappresentate anche in modo diverso: esistono dati di tipo categorico, sia con più categorie che con
due sole categoria, dati che possono essere considerati come booleani e dati interi, come ad esempio l'età.
\\\\
\section{Immagini}
\section{Metadati}
