\chapter*{Conclusioni}
\chaptermark{Conclusioni}
\addcontentsline{toc}{chapter}{Conclusioni}
In questo progetto di tesi si è focalizzata l'attenzione sull'uso dei reti neurali convoluzionali e di multilayer perceptron per analizzare ed effettuare 
previsioni usando un dataset fornito.
Il dataset si può suddividere nelle immagini, raffiguranti radiografie del torace, e nei metadati relativi alle immagini.
\\\\
In primo luogo si è compreso la composizione del dataset, si è osservato che le immagini avevano dimensioni varie e presentavano alcune differenze a livello 
di luminosità dell'immagine. Per tali motivi sono stati effettuati alcuni accorgimenti per migliorare la qualità delle immagini e ridimensionarle.
\\\\
In tal modo si è migliorata la qualità del dataset. Per procedere con il training della rete convoluzionale, si è dovuto prima creare immagini in cui si vedono solamente i polmoni.
Tale passaggio sfrutta la U-Net per individuare la collocazione dei polmoni nell'immagine.
\\\\
Una volta terminato questo studio sullle immagini si è passato al training vero e proprio della rete convoluzionale.
Per migliorare le prestazioni della rete sono stati applicate le tecniche di Fine Tuning e Transfer Learning.
Alla fine si è proceduto con l'effettuare dei training di test e i risultati ottenuti sono stati analizzati.
\\\\
Per l'implementazione dell'MLP si è deciso di usare dei layer che calcolano l'output partendo da un insieme di metadati presi dal dataset.
Anche in tal caso sono stati effettuati accorgimenti sul dataset, in modo da avere un suo sottinsieme completo, poiché erano presenti dati mancanti.
\\\\
In fine è stata creata un unica rete, composta dalle precedenti in modo che possa prendere in ingresso sia immagini che metadati e restituire un risultato più accurato.
\\\\
Una modifica interessante del lavoro effettuato è rendere il modello utilizzabile su dataset che presentano dati mancanti.
In alternativa si potrebbero comparare le prestazioni ottenute con reti differenti.