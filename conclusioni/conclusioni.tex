\chapter*{Conclusioni}
\chaptermark{Conclusioni}
\addcontentsline{toc}{chapter}{Conclusioni}
In questo progetto di tesi si è focalizzata l'attenzione sull'uso dei reti neurali convoluzionali e di percettrone multistrato (MLP) per analizzare ed effettuare 
previsioni usando un dataset fornito.
Il dataset si può suddividere nelle immagini, raffiguranti radiografie del busto, e nei metadati relativi alle immagini. Per comprendere come creare una rete neurale in grado di prendere in ingresso 
sia le immagini che i metadati relativi, si è partito studiando i due casi in maniera separata.
\\\\
Lo studio delle immagini e di come la relativa rete le processa risulta più complesso, per tale motivo si è partiti con l'affrontare tale 
problematica.
In primo luogo si è compreso la composizione del dataset e si è osservato che le immagini avevano dimensioni varie e presentavano alcune differenze a livello 
di luminosità dell'immagine. Per tali motivi sono stati effettuati alcuni accorgimenti per migliorare la qualità delle immagini e ridimensionarle.
Tali passaggi sono :
\begin{itemize}
    \item Verificare se l'immagine presentava fosse stata posta in negativo,
    \item Correggere la luminosità dell'immagine e equalizzare i colori usando opencv,
    \item Ridimensionare l'immagine aggiungendo il padding per evitare di modificare l'aspect-ratio dell'immagine.
\end{itemize}
Alla fine di tali passaggi si è ottenuto un dataset migliore. Per procedere con il training della rete convoluzionale, si è dovuto prima 
effettuare un altro passaggio riguardante al creazione di immagini in cui si vedono solamente i polmoni. Tale passaggio è composto da:
\begin{itemize}
    \item Creazione delle maschere mediante l'uso della U-Net,
    \item Creazione delle bounding-box usando le maschere trovate,
    \item Ridimensionamento ed eventuale padding finale dell'immagine.
\end{itemize} 
La rete usata come U-Net è EfficientNet e sono stati usati dei pesi preallenati per accellerare il processo e per ottenere maschere migliori.
\\\\
Una volta termimnato questo studio sullle immagini si è passato al training vero e proprio della rete convoluzionale.
La rete usata è stata composta usando come base MobileNetV2. Si è deciso anche qui di usare dei pesi preallenati, ma in questo caso si è dovuto 
effettuare un pasasggio di Fine tuning per adattare i pesi al dataset usato. A tale rete sono stati aggiunti layer per ridimensionare correttamente l'output ed ottenere 
dei valori binari.
Per allenare tali layer si è dovuto effettuare il Transfer Learning. 
Alla fine si è proceduto con l'effettuare dei training di test e i risultati ottenuti sono stati analizzati.
\\\\
Per l'implementazione dell'MLP si è deciso di usare un insieme di layer che calcolano l'output partendo da un insieme di metadati presi dal dataset.
Anche in tal caso sono stati effettuati accorgimenti sul dataset, in modo da avere un sottinsieme completo del dataset, poiché erano presenti dati mancanti.
Oltre a ciò, essendo il dataset suddiviso in training e test set, si è dovuto scegliere un sottinsieme comune ai due dataset.
\\\\
Ottenuto il dataset finale si è proceduto col normalizzare le varie tipologie di dati presenti in dati binari, ad esempio trasformando dati categorici, con più di due categorie, in dati binari, oppure 
trasformando dati interi in dati binari.
\\\\
A seguito della creazione dell'MLP si è proceduto, in fine, con la creazione di una rete unica, composta dalle precedenti in modo 
che possa prendere in ingresso sia immagini che metadati e restituire un risultato più accurato. Sono stati, anche in questo caso, effettuati 
dei training e valutati i risultati.
\section{Sviluppi Futuri}