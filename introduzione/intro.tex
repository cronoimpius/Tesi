%% motivazione
Il machine learning ha vari campi applicativi.
Uno di questi è quello di riconoscere dei pattern in modo che tali algoritmi possano apprendere e fare predizioni su un insieme di dati.

Un settore sul quale gli algoritmi di machine learning sono molto applicati è quello della medicina.
In questo documnento si tratterà di come usare gli algoritmi di machine learning per fare predizioni su un dataset che contiene dati relativi a pazienti affetti da Covid-19.

Esistono vari modi per analizzare i dati e fare apprendere la rete neurale in modo che le predizioni da essa effettuate abbiano senso.
Tali metodologie possono variare per via dei dati che compongono il dataset, ma anche dal tipo di previsioni che la rete deve effettuare.

L'algoritmo usato consiste in una rete neurale artificiale.
Tale rete è composta da vari neuroni, i quali hanno il compito di prendere dei dati in input, apprenderne le caratteristiche principali e sulla base di ciò effettuare le previsioni.

I neuroni che si occupano dell'apprendimento sono organizzati in vari layer nascosti. Ogni layer è composto da uno o più neuroni che interagiscono tra di loro. 

Come già accennato in base al tipo di dato da studiare si può prediligere una tipologia di rete ad un'altra.
Nel caso di questa tesi si è scelto di usare una rete neurale convoluzionale (CNN-\emph{Convolutional Neural Network}) per apprendere dalle immagini e una MLP ( \emph{Multi-Layer Perceptron} ) per gestire i metedati relativi alle immagini.

%% contributi tesi

%% organizzazione tesi

Questo elaborato di tesi inizia col trattare la struttura relativa al dataset considerato ed usato per allenare le 
reti neurali, descrivendo le varie tecniche usate per ovviare ai vari problemi che sono stati riscontrati durante il lavoro.
Successivamente si procede col dettagliare le modalità con cui sono state implementate ed allenate le reti usate, fornendo le motivazioni
di tali scelte ed i grafici per confrontare i risultati ottenuti. 