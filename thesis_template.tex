\documentclass[a4paper,11pt,oneside]{book}

\usepackage[utf8]{inputenc}
\usepackage[T1]{fontenc}
\usepackage[italian]{babel}
\usepackage{amsfonts,amsmath,amssymb,amsmath,amsthm,color}
\usepackage{graphicx}
\usepackage{float}
\usepackage{emptypage}
\usepackage{titlesec}
\usepackage{algorithm,algcompatible}
\usepackage{tabularx}
\usepackage[nottoc]{tocbibind}
\usepackage{subcaption}
\usepackage[backend=bibtex,style=ieee,sorting=none]{biblatex}
\usepackage[hidelinks]{hyperref}
\usepackage{filecontents}
\usepackage{csquotes}
\usepackage[font=small,labelfont=bf]{caption} % caption belle alle figure 
%per fare le tabelle dei csv belle
\usepackage[usenames,dvipsnames]{xcolor}
\usepackage{tcolorbox}
\usepackage{tabularx}
\usepackage{array}
\usepackage{colortbl}
\usepackage{booktabs}
\usepackage{makecell}
\tcbuselibrary{skins}


\newcolumntype{Y}{>{\raggedleft\arraybackslash}X}

\tcbset{tab1/.style={fonttitle=\bfseries\large,fontupper=\normalsize\sffamily,
colback=yellow!10!white,colframe=red!75!black,colbacktitle=Salmon!40!white,
coltitle=black,center title,freelance,frame code={
\foreach \n in {north east,north west,south east,south west}
{\path [fill=red!75!black] (interior.\n) circle (3mm); };},}}

\tcbset{tab2/.style={enhanced,fonttitle=\bfseries,fontupper=\normalsize\sffamily,
colback=gray75!10!white,colframe=black!50!black,colbacktitle=gray!40!white,
coltitle=black,center title}}

%Stile per l'indice
\renewcommand{\chaptermark}[1]{\markboth{#1}{}}
\renewcommand{\tocetcmark}[1]{\markboth{#1}{}}
\renewcommand{\sectionmark}[1]{\markright{\thesection\ #1}}

% Stile dei titoli dei capitoli
\definecolor{gray75}{gray}{0.75}
\newcommand{\hsp}{\hspace{20pt}}
\titleformat{\chapter}[hang]
{\Huge\bfseries}
{\thechapter\hsp\textcolor{gray75}{|}\hsp}
{0pt}
{\Huge\bfseries}

% Bibliografia
\bibliography{./bibliografia/bibliografia} % put your bibliography here

\setcounter{tocdepth}{1}
\raggedbottom

\graphicspath{{./immagini/}}

\begin{document}
	


%%%%%% First Page %%%

\pagestyle{myheadings}


\thispagestyle{empty}  
                                               
\begin{center}                                                            
    \vspace{2mm}
    {\large ALMA MATER STUDIORUM -- UNIVERSIT\`A DI BOLOGNA} \\  
                         
      \vspace{2mm}
\end{center}

\begin{center}
      \vspace{5mm}
      {\large \uppercase{Scuola di Ingegneria e Architettura}} \\
        \vspace{5mm}
       {\large Dipartimento di Informatica\\
       Scienza e Ingegneria }\\
   		{\large DISI}\\
        \vspace{5mm}
      {\Large \bf Corso di Laurea in Ingegneria Informatica}\\
      \vspace{5mm}
      %{ \textbf{Master Thesis}\\ in\\ \textit{Distributed Control Systems}}\\
      \vspace{5mm}
      {\LARGE\bf Classificazione della severità di patologia da Covid-19 mediante Transfer Learning su dataset eterogeneo} \\                
      \vspace{15mm}
      
      % table??
		\begin{tabularx}{\textwidth} 
      { 
				>{\raggedright\arraybackslash}X 
				>{\raggedleft\arraybackslash}X }
				{\large Candidato: } & {\large Relatore:} \\[3mm]
				{\large \itshape Gianmiriano Porrazzo  } & {\large \itshape prof. Andrea Camisa} \\[3mm]
				& {\large Correlatori: } \\[3mm] 
				& {\large \itshape dott. Andrea Testa} \\
        & {\large \itshape prof. Giuseppe Notarstefano}
    \end{tabularx}
      \vfill
      {\large Anno Accademico \\ \itshape 2021--2022} \\
      \vspace{5mm}
      {\large Sessione \\ \itshape II}
\end{center}


\vfill

\newpage
\thispagestyle{empty}



%%%%%%FRONTESPIZIO%%%%%%



%%%%%% INDICE %%%%%%%%%%

\tableofcontents

% elenco figure
\phantomsection
\listoffigures

%%%%%% ABSTRACT %%%%%%%%%%


%\chapter*{Abstract}
%\addcontentsline{toc}{chapter}{Abstract}
%
L'uso di tecniche di machine learnig è molto
%%%%%%% BODY %%%%%%%%%

\chapter*{Introduzione}
\addcontentsline{toc}{chapter}{Introduzione}
%% motivazione
Il machine learning ha vari campi applicativi.
Uno di questi è quello di riconoscere dei pattern in modo che tali algoritmi possano apprendere e fare predizioni su un insieme di dati.
\\\\
Un settore nel quale gli algoritmi di machine learning sono molto applicati è quello della medicina.
In questo documnento si tratterà di come usare gli algoritmi di machine learning per fare predizioni su un dataset che contiene dati relativi a pazienti affetti da Covid-19.
\\\\
Esistono vari metodologie per analizzare i dati e fare apprendere la rete neurale in modo che le predizioni da essa effettuate abbiano senso.
Tali metodologie possono variare per via dei dati che compongono il dataset, ma anche dal tipo di previsioni che la rete deve effettuare.
\\\\
L'algoritmo usato consiste in una rete neurale artificiale.
Tale rete è composta da vari neuroni, i quali hanno il compito di prendere dei dati in input, apprenderne le caratteristiche principali e sulla base di ciò effettuare le previsioni.
\\\\
I neuroni che si occupano dell'apprendimento sono organizzati in vari layer nascosti. Ogni layer è composto da uno o più neuroni che interagiscono tra di loro. 
\\\\
Come già accennato in base al tipo di dato da studiare si può prediligere una tipologia di rete ad un'altra.
Nel caso di questa tesi si è scelto di usare una rete neurale convoluzionale (CNN-\emph{Convolutional Neural Network}) per apprendere dalle immagini e una MLP ( \emph{Multi-Layer Perceptron} ) per gestire i metedati relativi alle immagini.
\\\\
%% contributi tesi
Il dataset usato nell'elaborato, per allenare le reti neurali, contiene immagini raffiguranti radiografie del torace 
di pazienti affetti da Covid-19 e altri dati relativi al loro stato di salute all'arrivo in ospedale. Per tale motivo 
si è cercato di suddividere le reti da creare in base agli input che prende.
\\\\
Il primo passo per implementare la CNN è comprendere meglio come il dataset delle immagini è composto, per tale motivo 
si è effettuata un'analisi accurata di tali immagini.
Da tale analisi è emerso che le immagini sono molto eterogenee e che necessitano di alcuni accorgimenti prima di poter essere 
date in input alla rete. Questa serie di passi prende il nome di preprocessing delle immagini.
Nel caso considerato tale preprocessing prevede: regolazione di alcune caratteristiche dell'immagine (contrasto, luminosità, ...),
riconoscimento dei polmoni mediante una rete apposita (U-Net) e la creazione dell'immagine finale mediante l'uso di bounding box.
\\\\
Per quanto riguarda le ultime due fasi sono effettuate partendo dall'uso di una rete convoluzionale, che ha come scopo 
quello di eseguire la segmentazione dell'immagine per trovare la posizione dei polmoni, dalla quale otteniamo delle maschere, ovvero 
immagini in cui si nota il contorno del soggetto dell'immagine. Da tali maschere si ricava poi la bounding box, il più piccolo
riquadro contenente il soggetto d'interesse.
Grazie a questi due artefatti si ottene l'immagine finale, di qualità migliore rispetto all'originale.
In tal modo allenare la rete convoluzionale che deve effettuare previsioni risulta più semplice poiché tali immagini, sono alla fine del
processo appena descritto, ridimensionata ad una dimensione e sono incentrate sui polmoni. 
\\\\
Ottenute le immagini si procede con l'allenare la rete neurale convoluzionale e, in base ai risultati ottenuti, si decide se applicare 
tecniche per migliorare l'apprendimento della rete. Nel nostro caso sono state usate la Data Augmentation, il Transfer Learning e il Fine Tuning.
\\\\
La Data Augmentation è una tecnica usata per modificare immagini presenti nel dataset applicando alcune trasformazioni ad esse, in modo 
che la rete non si abitui alle immagini e restituisca previsioni più accurate.
\\\\
Il Transfer Learning consiste nell'uso di una rete pre-allenata per affrontare problemi correlati a quelli per cui era stata allenata.
In tal modo la rete, in seguito ad opportuni accorgimenti e nuovi allenamente, risulta esssere più precisa.
Uno di tali accorgimenti consiste nel fine tuning che permette di allenare gli ultimi layer della rete, ovvero quelli relativi alla classificazione,
in modo da essere performanti sul nuovo dataset.
\\\\
Per quanto riguarda la gestione dei metadati, si parte, anche in questo caso, con l'analizzare il dataset che ci viene fornito.
A seguito di tale analisi si verifica che il dataset non è completo, ovvero presenta dati mancanti. Per tale motivo si decide di 
trovare un sottinsieme del dataset contenente il maggior numero di dati, in modo che questo sia completo.
A questo punto si procede con la creazione del percettrone multistrato, ovvero una piccola rete neurale composta da pochi layer con il compito 
di gestire gli input e di classificare gli output nel modo corretto.
Prima di passare i dati in ingresso al percettrone questi devono essere convertiti in un formato comune per via del fatto che, come si vedrà 
più nel dettaglio, i dati scelti sono di varia natura.
\\\\
Terminata tale fase si è pronti per concatenare le reti create e gestire i dati mediante un'unica rete che gestisce sia le immagini che gli altri dati 
relativi al paziente.
\\\\
%% organizzazione tesi
Questo elaborato di tesi inizia col trattare la struttura relativa al dataset considerato ed usato per allenare le 
reti neurali, descrivendo le varie tecniche usate per ovviare ai vari problemi che sono stati riscontrati durante il lavoro.
Successivamente si procede col dettagliare le modalità con cui sono state implementate ed allenate le reti usate, fornendo le motivazioni
di tali scelte ed i grafici per confrontare i risultati ottenuti.
Si inizia col trattare la rete che gestisce le immagini come input (CNN), successivamente si descrive la tecnica usata per creare 
un percettrone multistrato in base ai dati che verranno passati in input ed, infine, sono state combinate queste due reti al fine 
di crearne una più accurata nelle previsioni poiché prende in input sia i dati che le immagini.  

%\markboth{}{INTRODUZIONE}
	
%	\section*{Motivazioni}
%	\addcontentsline{toc}{section}{Motivazioni}
	
%  (articoli e references)
		
%	\section*{Letteratura}
%	\addcontentsline{toc}{section}{Letteratura}
	
	
%	\section*{Contributi}
%	\addcontentsline{toc}{section}{Contributi}
	
%	\section*{Organizzazione}

%%%%%%% CAPITOLI %%%%%%%%%

\chapter{Conoscenze preliminari}
\label{ch:Conoscenze Preliminari}
\section{Cosa s'intende per rete neurale e come si allenano}
Le reti neurali, spesso chiamate anche reti neurali artificiali o simulate, sono algoritmi con nomi e strutture 
ispirate al cervello umano ed hanno lo scopo di ricreare il modo in cui i neuroni umani comunicano.
Le reti artificiali sono, nella maggiorparte dei casi, sistemi adattivo, ovvero possono cambiare
la propria struttura in base alle informazioni che circolano all'interno della rete.
Una rete neurale non è altro che una funzione matematica. Una rete neurale solitamente è definita 
da una serie di neuroni (cos'è un neurone \ref{neuroni}) connessi tra di loro.
% strutturazione delle reti neurali
Solitamente le reti artificiali sono composte da più livelli: un livello relativo alla gestione dell'input,
uno o più livelli nascosti ed un livello per la gestione degli output.
Ogni livello è composto da nodi o neuroni artificiali.
La connessione tra ognuno di questi nodi avviene mediante l'uso di sogli e pesi. Se, ad esempio, 
l'output generato da un nodo è al di sopra di una soglia specificata, il nodo viene attivato in modo 
da inviare dati al livello successivo della rete, altrimenti non vengono inviati dati.
\\\\
Ogni neurone è composto da dati di input, pesi, una soglia (detta anche bias) e dai dati restituiti in output.
una volta che le dimensioni dei dati in ingresso vengono determinati, si assegnano anche i pesi.
I pesi sono molto importanti, poiché aiutano a determinare l'importanza di ogni variabile in ingresso 
al neurone: pesi più grandi indicano che la varabile ha maggior importanza nella determinazione dell'output,
pesi minori indicano che la variabile è meno importante per la determinazione dell'output.
\\\\
Altro valore importante è quello relativo alla soglia, poiché se il valore determinato come output supera la soglia, 
il nodo viene attivato. In tal modo vengono passati i dati al neurone successivo.
Così l'output di un nodo diventa l'input del successivo.
\\\\
All'inizializzazione della rete, i pesi vengono scelti in maniera casuale, portando la rete 
ad avere scarsi risultati. Allenare una rete significa passare da una rete poco performante ad 
una con un'accuracy elevata. Ciò è reso possibile dal fatto che possiamo modificare la funzione 
che gestisce la rete, facendo degli accorgimenti sui pesi.
\\\\
Essendo la rete neurale una funzione, quando alleniamo la rete per avere prestazioni migliori,
ciò che facciamo è minimizzare la funzione di perdita (loss function). Si decide di minimizzare tale 
funzione perché, in seguito a vari esperimenti, è stato verificato che ottimizzare tale funzione 
è più semplice. Esistono vari  algoritmi per ottimizzare funzioni, questi possono essere basati sul 
gradiente o meno.
\\\\
In virtù del fatto che l'obiettivo è trovare il minimo della funzione, vengono effettuati tentetivi
ad ogni iterazione dell'allenamento. Un parametro fondamentale in questo processo è il \emph{learning rate}.
Tale parametro indica la grandezza del salto che viene effettuato ad ogni iterazione per avvicinarsi al 
minimo della funzione. Per tale motivo se tale valore è troppo elevato, saranno effettuati salti 
troppo grandi con il rischio di superare il minimo e conseguente divergenza dell'algoritmo.
Se, al contrario si usa un learning rate troppo piccolo i tempi per trovare il minimo potrebbero aumentare 
ed il minimo a cui si converge potrebbe essere un minimo locale.

\section{Cosa sono i neuroni nelle delle reti neurali}
\label{neuroni}
Un neurone non è altro che una funzione matematica. Tale funzione ha un input, di dimensione 
determinata, e restituisce un output. Sono, inoltre, presenti 
dei coefficienti moltiplicativi fondamentali per determinare l'output : i pesi.
\\\\
I pesi diversificano i neuroni e vengono determinati durante l'allenamento della rete, ovvero la 
fase in cui si regolano la rete.
\\\\
La funzione che descrive un neurone può essere divisa in due parti: combinazione lineare di 
input e pesi preceduta da una funzione non lineare. La funzione non linear più usata è ReLU (Rectified 
Linear Unit).
\begin{equation*}
    ReLU(x)=max(x,0)
\end{equation*}
Tale funzione risulta essere pari a zero se $x$ è minore di zero e restituisce $x$ se tale valore 
è maggiore di zero.
\\\\
Infine è utile ricordare che il valore restituito dal neurone è indicato col termine attivazione, 
poiché determina se il neurone verrà attivato, passando i dati al neurone successivo nella catena, o meno.
\section{A cosa serve la segmentazione delle immagini e cos'è una U-Net}
Nel caso di classificazione delle immagini la rete neurale assegna un'etichetta (o classe) ad ogni 
immagine in input. Se si vuole conoscere la forma di un oggetto presente nell'immagine, quali pixel appartengono
a quale oggetto o cose simili, si deve assegnare una classe ad ogni pixel dell'immagine e non all'immagine intera.
Tale attività è conosciuta con il nome di segmentazione. 
Grazie a tale tecnica tutti i pixel appartenenti allo stesso oggetto avranno la stesssa etichetta e
sarà più facile compiere analisi più dettagliate sull'immagine.
\\\\
Nel caso del database usato, si è interessati ai soli polmoni del paziente per tale motivo è utile 
effettuare segmentazione in modo da distinguere ciò che fa parte dal polmnone dal resto.
\\\\
Esistono vari modi per effettuare segmentazione delle immagini tra cui:
\begin{itemize}
    \item Segmentazione basata su soglia
    \item Sementazione basata sulle regioni
    \item Segmentazione basata sui margini
    \item Segmentazione basata su gruppi
    \item Segmentazione basata su reti artificiali
\end{itemize}
Nel caso d'interesse si è scelto di usare la segmentazione basata su reti artificiali.
Per tale motivo è stato necessario usare una rete neurale apposita per effettuare segmentazione.
Tale tipologia di rete è conosciuta con il nome di U-Net. Le U-Net sono reti neurali convoluzionali specializzate
nella segmentazioni di immagini biomediche.
\\\\
La U-Net è stata usata per creare delle maschere, ovvero immagini che separano ciò che è all'interno del polmoni 
con ciò che non appartiene ai polmoni. Dunque la previsione che tale rete produce è un'immagine con la 
stessa dimensione dell'immagine iniziale.
\\\\
Dalla maschera creata sarà poi più semplice effettuare ulteriori analisi e modifiche 
sull'immagine iniziale.
\pagebreak
\section{Codifica one-hot per dati categorici}
La codifica one-hot risulta essere molto utile quando si è in presenza di dati categorici. Tale tipologia di dati
contengono etichette relative al dato invece che valori numerici: ogni etichetta solitamente rappresenta 
categorie diverse.
Il problema legato a tale tipologia di dati è che una rete neurale non riesce ad operare direttamente c
sulle etichette dei dati, poiché richiedono che tutte le variabili in ingresso ed uscita siano numeriche.
Per tale motivo si necessita i dati categorici devono essere convertiti in numerici.
La conversione da dati categorici a numerici prevede due passi:
\begin{itemize}
    \item Integer Encoding
    \item One-Hot Encoding
\end{itemize}
Il primo passo prevede che ad ogni categoria venga assegnato un valore numerico, questo potrebbe in 
alcuni casi essere sufficiente per convertire i dati.
Il secondo passo prevede la conversione dei valori numerici in valori binari.
Ciò che solitamente accade è che vengono scelti tanti bit quanti sono le categorie da rappresentare 
e, successivamente, ogni categoria verrà rappresentata da una serie di bit posti come 0 ed un unico bit 
posto ad 1.
\\\\
Tale processo viene usato all'interno dell'elaborato di tesi al momento dello studio del dataset eterogeneo 
e la realizzazione del percettrone.
\section{Strumenti usati}
Esistono vari framework per lo sviluppo di reti neurali, uno dei più usati è TensorFlow \cite{tsf}. 
\\
TensorFlow presenta modelli già creati e allenati, i cui pesi possono essere presi per sfruttare tale modello.
Insieme all'uso di altre librerie come Keras l'implementazione di una rete neurale che gestisca i dati a disposizione 
è molto semplificato.
\\
La rete di base usata per partire con l'analisi delle immagini è la MobileNetV2. Tale rete è una CNN che, prendendo come input delle 
immagini.
A tale rete, tuttavia, sono state apportate delle modifiche per via delle dimensioni dell'input e della tipologia di dato 
che si vuole ottenere come previsione.
\\
Oltre a questa tipologia di rete si è sfruttata anche una U-Net, al fine di omogeneizzare tutte le immagini che vengono date come input della MobileNetV2.
La U-Net usata è EfficientNet, con dei pesi preallenati per riconoscere immagini simili a quelle presenti nel dataset di riferimento.


\chapter{Dataset}
\label{ch:Dataset}

*inserire link all'hackaton e immagini dataset*

Il dataset usato per effettuare lo studio è stato preso da un hackaton riguardante la creazione di una rete neurale in grado di apprendere 
da immagini relative a radiografie di pazienti affetti da covid in modo da predirre la gravità della prognosi.
\\
Dunque il dataset usato è formato da immagini raffiguranti radiografie dei polmoni, le quali tuttavia non sono omogenee. Per tale motivo 
si è dovuto effettuare un passaggio preliminare in modo da rendere le immagini tutte della stessa dimensione, ma non distorcendole.
\\
Oltre alle immaggini sono presenti anche un insieme di metadati riguardanti lo stato in cui il paziente è entrato all'ospedale e l'anamnesi dello stesso.

\chapter{Implementazione: CNN}
\label{ch:CNN}
\section{Creazione rete neurale}
Lo sviluppo della rete neurale convoluzionale, per effettuare previsioni partendo dalle immagini, è basata sulla rete 
MobileNetV2. L'uso di tale rete come base consente di usare dei pesi preallenati in modo da rendere il processo di training sul dataset più efficiente.
\\\\
Tuttavia tale processo necessita di alcune fasi aggiuntive per fare in modo che la rete riesca a prendere i dati della dimensione esatta 
e restituire delle previsioni nell'insieme desiderato.
\\\\
Tali operazioni sono:
\begin{itemize}
    \item Inserire la dimensione delle immagini nel layer di input
    \item Effettuare fine tuning e transfer learning per poter usare i pesi preallenati 
    \item Inserire dei layer finali per ottenere degli output significativi
\end{itemize}

La dimensione scelta delle immagini è (224x224), dunque la rete prende in ingresso degli array di dimensione 224x224x3 (dove quest'ultimo indica l'uso dei colori RGB).
Per tale motivo il layer di input deve accettare tale dimensione.
\\\\
I pesi preallenati che si sono scelti sono pesi allenati su ImageNet. ImageNet è un dataset di immagini suddivise in 1000 classi. Il dataset su cui si deve svolgere il training, tuttavia, possiede 
unicamente una classe, per via del fatto che abbiamo trasformato il valore della prognosi da una string ad un valore binario. 
Per cui l'immagine può appartenere o meno a tale classe.
\\\\
Al fine di poter usare tali pesi per effettuare il training, si necessita dunque di ulteriori passaggi: 
\begin{itemize}
    \item Transfer Learning
    \item Fine tuning
\end{itemize}

\section{Transfer Learning}

Il transfer learning consente di sfruttare la rete già allenata a risolvere problemi diversi, ma comunque correlati con quello di interesse.
Nel caso considerato tale tecnica permette di usare una rete allenata per prevedere l'appartenenza di una immagine ad una delle 1000 classi di ImageNet
per creare una rete in grado di classificare le immagini del dataset in una unica classe.
\\\\
Per sfruttare la rete allenata, congeliamo lo stato dei layer di classificazione, al fine di non alterarli, e settiamo la rete come non allenabile.
In tal modo si è ottenuto un nuovo modello basato sulla MobileNetV2.
\\\\
Ora si presenta un problema relativo alla classificazione. Per ovviare al fatto che tale operazione sarà fatta per una sola classe, si necessita dell'inserimento di altri layer alla fine del modello precedentemente ottenuto.
\\\\
Tali layers sono:
\begin{itemize}
    \item un GlobalAveragePooling2D()
    \item due Dense() 
\end{itemize}  

Il primo layer serve per via del fatto che allo stato attuale la rete produce un output multidimensionale e, per ottenere previsioni formate da un singolo vettore 
della dimensione prevista, usiamo tale layer, il quale genera previsioni basate sul blocco multidimensionale e facendone una media.
\\\\
Il primo Dense layer viene usato per generare l'output prodotto dalle immagini dei polmoni. Tale layer è formato da 100 neuroni ed ha come funzione di attivazione ReLu.
\\\\
Il secondo Dense layer svolge il lavoro di classificazione vero e proprio. Per via del fatto che si è scelto di usare una label binaria, ovvero classifichiamo su un'unica classe, tale 
layer è composto da un solo neurone, ed ha come funzione di attivazione Sigmoid.
\\\\
Una volta creati i layers, per completare la fase di transfer learning, si deve compilare il modello finale, ottenuto aggiungendo i nuovi layers.
\\\\
Ora la rete è pronta per una prima fase di training. In tale fase sono stati usati i seguenti parametri per il training:
\begin{itemize}
    \item Loss function: Adam
    \item Metrica: accuracy
    \item Epoche 150
    \item Batch size: 30
\end{itemize}
$\\$
Per quanto concerne il valore del learning rate sono si è deciso di effettuare prove sia con $10^{-3}$ che con 
$10^{-4}$.
I risultati ottenuti dal training sono espressi nei seguenti grafici.

\begin{figure}[h]
    \centering
    \includegraphics[width=12cm]{./10-3_150.png}
    \label{ 10^{-3} }
    \caption{Test effettuato usando come learning rate $10^{-3}$}
\end{figure}

\begin{figure}[ht]
    \centering
    \includegraphics[width=12cm]{./10-4_150.png}
    \label{10^{-4}}
    \caption{Test effettuato usando come learning rate $10^{-4}$}
\end{figure}
\vspace{1000mm}

\section{Fine Tuning}

Per procedere con la fase di fine tuning, si deve rendere nuovamente trainabile il modello creato, 
in modo tale che i pesi possano essere allenati considerando i nuovi layers.
Così facendo si permette ai pesi di regolarsi sul dataset d'interesse, partendo da quello su cui sono stati inizialmente allenati.
\\\\
Essendo gli ultimi layers di una rete inutili in termini di classificazione, possiamo decidere di 
congelarli, ovvero non considerarli durante il training, in modo da risparmiare risorse.
Effettuato tale passaggio si può procedere con la compilazione del modello ottenuto e procedere con il training.
\\\\
I parametri relativi al training in questa fase sono:
\begin{itemize}
    \item Loss function: Adam
    \item Metrica: accuracy
    \item Epoche 30
    \item Batch size: 30
\end{itemize}
$\\$
Di seguito sono riportati i risultati ottenuti a seguito del transfer learning e fine tuning, sempre considerando 
i due valori scelti per il learning rate.

\begin{figure}[htp]
    \centering
    \includegraphics[width=10cm]{./10-3_150_2.png}
    \label{ 10^{-3} }
    \caption{Test effettuato usando come learning rate $10^{-3}$}
\end{figure}

\begin{figure}[htp]
    \centering
    \includegraphics[width=10cm]{./10-4_150_2.png}
    \label{10^{-4}}
    \caption{Test effettuato usando come learning rate $10^{-4}$}
\end{figure}
\vspace{1000mm}
\section{Overfit e Data Augmentation}
Dai risultati ottenuti risulta evidente la presenza di overfitting nella rete.
Tale problema si è potuto notare per via del fatto che, anche a seguito del fine tuning, la validation accuracy non tende 
ad aumentare.
\\\\
L'overfit può essere causato dal fatto che la rete si abitui agli input del training set, o al 
fatto che questi siano molto simili tra di loro e risponda in maniera casuale a nuovi dati. Per tale motivo si è scelto di usare la data augmentation come tecnica risolutiva.
La data augmentation consiste nel prendere le immagini del dataset, effettuare delle modifiche ad esse (come rotazioni), per 
creare una immagine nuova, in modo da introdurre diversità all'interno del dataset. 
\\\\
Per implementare tale tecnica si è optato per l'uso del ImageDataGenerator(). L'adozione di tale funzionalità presente in 
TensorFlow è dovuta al fatto che consente di creare immagini partendo dal dataset iniziale, applicando trasformazioni casualmente, scegliendo tra 
quelle selezionate.
\\\\
Le trasfromazioni che sono state scelte, in base alla possibilità che la rete consideri anche le nuove immagini come 
valide, sono la rotazione verticale ed orizontale, specificando anche il range di angolo in cui effettuare tale rotazione.
La scelta di queste trasformazioni è stata validata anche osservando le immagini prodotte e riflettendo sul fatto che fossero 
sensate come input della rete.
\begin{figure}[h]
    \centering
    \begin{subfigure}{.45\textwidth}
        \centering
        \includegraphics[width=.95\linewidth]{augmented_ex.png}  
        %\caption{}
        %\label{SUBFIGURE LABEL 1}
    \end{subfigure}
    \begin{subfigure}{.45\textwidth}
        \centering
        \includegraphics[width=.95\linewidth]{augmented_ex_1.png}  
        %\caption{}
        %\label{SUBFIGURE LABEL 2}
    \end{subfigure}
    \begin{subfigure}{.45\textwidth}
        \centering
        \includegraphics[width=.95\linewidth]{augmented_ex_2.png}  
        %\caption{}
        %\label{SUBFIGURE LABEL 3}
    \end{subfigure}
    \begin{subfigure}{.45\textwidth}
        \centering
        \includegraphics[width=.95\linewidth]{augmented_ex_3.png}  
        %\caption{}
        %\label{SUBFIGURE LABEL 4}
    \end{subfigure}
    \caption{Immagini a seguito dell'augmentation, con relativa label}
    \label{Augmentation}
\end{figure}
\\\\
È importante sottolineare che l'augmentation viene effettuata unicamente sul training set.
\\\\
Per verificare se la data augmentation ha avuto effetto si procede con dei training della rete.
Tali training sono stati effettuati mantenendo gli stessi parametri precedentemente usati.

\begin{figure}[h]
    \centering
    \begin{subfigure}{.75\textwidth}
        \centering
        \includegraphics[width=.95\linewidth]{lr10-31_r.png}  
        %\caption{}
        %\label{SUBFIGURE LABEL 1}
    \end{subfigure}
    \begin{subfigure}{.75\textwidth}
        \centering
        \includegraphics[width=.95\linewidth]{lr10-32_r.png}  
        %\caption{}
        %\label{SUBFIGURE LABEL 2}
    \end{subfigure}
    \caption{Training effettuato con learning rate pari a $10^{-3}$ a seguito dell'augmentation}
    \label{Training Augmentation}
\end{figure}

\begin{figure}[h]
    \centering
    \begin{subfigure}{.75\textwidth}
        \centering
        \includegraphics[width=.95\linewidth]{lr10-41_r.png}  
        %\caption{}
        %\label{SUBFIGURE LABEL 3}
    \end{subfigure}
    \begin{subfigure}{.75\textwidth}
        \centering
        \includegraphics[width=.95\linewidth]{lr10-42_r.png}  
        %\caption{}
        %\label{SUBFIGURE LABEL 4}
    \end{subfigure}
    \caption{Training effettuato con learning rate pari a $10^{-4}$ a seguito dell'augmentation}
    \label{Training Augmentation}
\end{figure}
\chapter{Design e training di una rete neurale con dati eterogenei in ingresso}
\label{ch:MLP}
\section{Preprocessing dataset}
I risultati ottenuti con la CNN possono essere migliorati o resi più attendibili con l'utilizzo
di altre tipologie di dati. Questi sono le informazioni relative al paziente al momento dell'arrivo in 
ospedale.
\\\\
Tali dati sono di vario tipo, quelli usati dal dataset considerato sono: 
\begin{itemize}
    \item Categorico, esprimono l'appartenenza ad una o più categorie
    \item Booleano, esprimo se il paziente è affetto da una certa patologia o presenta alcuni 
    \item Numerici
\end{itemize}
In questa fase è importante che tutti i dati siano presenti per ogni paziente. Per tale motivo si è dovuto 
trovare un sottinsieme del dataset che presenti il maggior numero di dati. 
\\\\
Per fare ciò sono state eliminate le colonne che non avevano corrispondenze 
tra i pazienti, ovvero erano vuote oppure presentavano dati solo per pochi pazienti.
Al fine di ottenere un risultato migliore sono state eliminati anche i pazienti che non presentavano dati per 
la maggiorparte delle colonne del dataset.
\\\\
Effettuata tale operazione, si è proceduto col creare un unico dataset ottenuto dall'unione 
di training set e test set. Per fare ciò le colonne presenti tra i due devono essere le stesse.
Per tale motivo si è giunti ad un dataset formato dalle seguenti colonne:
\begin{itemize}
    \item Ospedale, dato categorico che rappresenta l'ospedale che ha accolto il paziente tra A,B,C,D,E e F 
    \item Età, dato numerico
    \item Sesso, categorico binario, ovvero maschio o femmina
    \item Tosse, binario
    \item Difficoltà respiratorie
    \item Numero di cellule bianche, dato numerico che indica la percentuale di globuli bianchi nel sangue
    \item Pressione sanguigna alta,  binario
    \item Diabete, binario
    \item Demenza, binario 
    \item BPCO (Broncopneumopatia cronica ostruttiva), binario 
    \item Cancro, binario 
    \item Malattia renale cronica, binario
\end{itemize}
Tali dati sono presenti per 946 pazienti del training set e 472 del test set, per cui abbiamo un dataset di 1218 pazienti.
Questo dataset sarà poi suddiviso in training, validation e test set, per effettuare l'allenamento della rete e per verificare la capacità di effettuare previsioni.
$\\\\$
\begin{tcolorbox}[tab2,tabularx={Y|Y|Y|Y|Y|Y|Y|Y|Y|Y},title=\text{Estratto del dataset dato dall'unione dei due di partenza},width=\textwidth, center=\textwidth]
    \centering
    \begin{tabular}{l|c|c|c|c|c|c|c|l}
        ImageFile & H. & Age & .... & Cough & WBC & Ic. & H.B.P. & Prognosis \\ \hline \hline
        ... & ... & ... & ... & ... & ... & ... & ... & ...\\
        P\_281.png & E & 68 &...  & 0 & 7.33  & 0 & 1 &  MILD   \\
        P\_544.png & F & 72 &...  & 0 & 9.6  & 0 & 1 &  SEVERE   \\
        P\_657.png & C & 83 & ... & 1 & 9 & 0 & 1 &  SEVERE   \\
        P\_1\_93.png & F & 66 & ... & 0 & 10 & 0 & 0 & SEVERE   \\
        P\_73.png & A & 48 & ...  & 1 & 9.13 & 0 & 0  & MILD  \\
        ... & ... & ... & ... & ... & ... & ... & ... & ...
    \end{tabular}     
\end{tcolorbox}
$\\\\$
Ora che si presenta il dataset completo, si procede col trasformare tutti i dati nello stesso formato, ovvero in valori binari \cite{ar}.
Partendo dai dati categorici, si procede usando la codifica one-hot. Tale procedura prevede che le categorie relative al dato vengano 
trasformate in una rappresentazione binaria, in cui ogni categoria è rappresentata da una serie di zeri ed un unico uno presente nella categoria
codificata.
Per quanto riguarda il sesso, essendo una categoria binaria, si può usare un valore binario, per cui un sesso sarà rappresentato da 0 e l'altro da 1.
L'ospedale è un dato che prevede sei categorie, per cui queste saranno rappresentate da sei bit. Ogni rappresentazione
presenterà cinque zeri ed un unico uno (es. 001000).
\\\\
A livello pratico tali trasfromazioni sono state implementate usando la libreria scikit-learn, in particolare le funzioni
MultiLabelBinarizer(), per l'ospedale, e LabelBinarizer() per il sesso. In tal modo si riesce a trasformare le categorie in array di 
bit, ovvero valori compresi in [0,1].
\\\\
Per gestire i valori numerici si sfrutta un'altra funzione di scikit-learn, ovvero MinMaxScaler().
Tale funzione scala i valori dati in input in un range specificato, nel caso in considerazione [0,1].
La trasformazione avviene mediante:
\begin{equation*}
    \begin{array}{l}
        X_{std} = \dfrac{(X - X.min(axis=0))}  {(X.max(axis=0) - X.min(axis=0))} \\\\
        X_{scaled} = X_{std} * (max - min) + min
    \end{array}
\end{equation*}
\\\\
I valori binari infine sono rimasti invariati, poichè esprimono il valore nel range desiderato.

\section{Struttura percettore multistrato}

Avendo ora convertito i dati del dataset in modo da essere in [0,1], si può iniziare a costruire la rete che deve fare previsioni 
prendendo come input tali dati. 
È importante notare che una volta effettuata la codifica one-hot della colonna relativa all'ospedale, si ottiene un vettore di dimensione 6 (una per ogni categoria) al 
posto dell'unica colonna che rappresenta il dato. Per tale motivo la dimensione dell'input della rete non è 
più 12 (in base al numero di colonne presenti inizialmente), ma 17.
\\\\
La rete dunque sarà formata da due layer:
\begin{itemize}
    \item un Dense() layer formato da 30 neuroni
    \item un Dense() layer formato da 5 neuroni
\end{itemize}
Il primo layer è colui che si occupa di ricevere anche gli input, per cui la dimensione degli input dovrà essere pari a 17.
Il secondo layer, invece, è formato da soli 5 layer e si occupa di effettuare una prima classificazione.
Entrambi i layer presentano come funzione di attivazione ReLu.
\\\\
Con tale rete, si riesce dunque ad effettuare il training relativo all'uso dei metadati.

\section{Creazione rete composta}
Per poter usare contemporaneamente le due reti neurali create, ovvero la CNN e la MLP, si necessita di 
un ulteriore passaggio.
Lo scopo di tale passaggio è avere una rete che accetti come input sia immagini che i metadati relativi, al fine
di ottenere una previsione più affidabile, poiché basata sull'uso di più dati.
\\\\
Per gestire gli output delle due reti questi vengono concatenati, in modo da ottenere un unico array di output.
Al fine di effettuare le predizioni su tale output si aggiunge, alla fine della rete, un ulteriore 
Dense layer, formato da un unico neurone, con funzione di attivazione Sigmoid e che prende un input della 
dimensione della combinazione degli output delle due reti. 
\\\\
Si può, infine, creare la rete che prende in input la combinazione degli input della rete, mentre l'output sarà determinato
dal nuovo layer inserito.
*immagine rete completa*
Ora si può dunque procedere con la compilazione della rete e con il training. I training sono stati effettuati usando i parametri usati 
nel caso della rete convoluzionale. 

\begin{figure}[b]
    \centering
    \includegraphics[width=\textwidth]{10_3_clin_giusto.png}
    \label{ 10^{-3} }
    \caption{Test effettuato usando come learning rate $10^{-3}$}
\end{figure}

\begin{figure}[htp]
    \centering
    \includegraphics[width=\textwidth]{10_4_clin_giusto.png}
    \label{10^{-4}}
    \caption{Test effettuato usando come learning rate $10^{-4}$}
\end{figure}
%\chapter{Risultati}
\label{ch:MLP}
\section{Risultati ottenuti}
*discussione sui risultati*


%%%%%%%%%% APPENDIX %%%%%%%%%%%%%%%%%%%

%\appendix
%\chapter{Appendice}
%%%%%%%%%% CONCLUSIONI %%%%%%%%%%%%%%%%%%%
\chapter*{Conclusioni}
\chaptermark{Conclusioni}
\addcontentsline{toc}{chapter}{Conclusioni}
In questo progetto di tesi si è focalizzata l'attenzione sull'uso dei reti neurali convoluzionali e di percettrone multistrato per analizzare ed effettuare 
previsioni usando un dataset fornito.
Il dataset si può suddividere nelle immagini, raffiguranti radiografie del torace, e nei metadati relativi alle immagini.
\\\\
In primo luogo si è compreso la composizione del dataset, si è osservato che le immagini avevano dimensioni varie e presentavano alcune differenze a livello 
di luminosità dell'immagine. Per tali motivi sono stati effettuati alcuni accorgimenti per migliorare la qualità delle immagini e ridimensionarle.
\\\\
In tal modo si è migliorata la qualità del dataset. Per procedere con il training della rete convoluzionale, si è dovuto prima creare immagini in cui si vedono solamente i polmoni.
Tale passaggio sfrutta la U-Net per individuare la collocazione dei polmoni nell'immagine.
\\\\
Una volta termimnato questo studio sullle immagini si è passato al training vero e proprio della rete convoluzionale.
Per migliorare le prestazioni della rete sono stati applicate le tecniche di Fine Tuning e Transfer Learning.
Alla fine si è proceduto con l'effettuare dei training di test e i risultati ottenuti sono stati analizzati.
\\\\
Per l'implementazione dell'MLP si è deciso di usare dei layer che calcolano l'output partendo da un insieme di metadati presi dal dataset.
Anche in tal caso sono stati effettuati accorgimenti sul dataset, in modo da avere un suo sottinsieme completo, poiché erano presenti dati mancanti.
\\\\
In fine è stata creata un unica rete, composta dalle precedenti in modo che possa prendere in ingresso sia immagini che metadati e restituire un risultato più accurato.
\\\\
Una modifica interessante del lavoro effettuato è rendere il modello utilizzabile su dataset che presentano dati mancanti.
In alternativa si potrebbero comparare le prestazioni ottenute con reti differenti.
%%%%%%%%%% Ringraziamenti %%%%%%%%%%%%%%
%%%%%%%%%% BIBLIOGRAPHY %%%%%%%%%%%%%%
\phantomsection
\printbibliography[heading=bibintoc]

%%%%%%%%%%%%%%%%%%%%%%%%%%%%%%%%%%%%%%

\end{document}